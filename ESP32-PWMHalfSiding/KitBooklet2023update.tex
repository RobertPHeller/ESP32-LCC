%%%%%%%%%%%%%%%%%%%%%%%%%%%%%%%%%%%%%%%%%%%%%%%%%%%%%%%%%%%%%%%%%%%%%%%%%%%%%
%
%  System        : 
%  Module        : 
%  Object Name   : $RCSfile$
%  Revision      : $Revision$
%  Date          : $Date$
%  Author        : $Author$
%  Created By    : Robert Heller
%  Created       : Sun Jan 1 13:48:18 2023
%  Last Modified : <230102.1007>
%
%  Description 
%
%  Notes
%
%  History
% 
%%%%%%%%%%%%%%%%%%%%%%%%%%%%%%%%%%%%%%%%%%%%%%%%%%%%%%%%%%%%%%%%%%%%%%%%%%%%%
%
%    Copyright (C) 2023  Robert Heller D/B/A Deepwoods Software
%			51 Locke Hill Road
%			Wendell, MA 01379-9728
%
%    This program is free software; you can redistribute it and/or modify
%    it under the terms of the GNU General Public License as published by
%    the Free Software Foundation; either version 2 of the License, or
%    (at your option) any later version.
%
%    This program is distributed in the hope that it will be useful,
%    but WITHOUT ANY WARRANTY; without even the implied warranty of
%    MERCHANTABILITY or FITNESS FOR A PARTICULAR PURPOSE.  See the
%    GNU General Public License for more details.
%
%    You should have received a copy of the GNU General Public License
%    along with this program; if not, write to the Free Software
%    Foundation, Inc., 675 Mass Ave, Cambridge, MA 02139, USA.
%
% 
%
%%%%%%%%%%%%%%%%%%%%%%%%%%%%%%%%%%%%%%%%%%%%%%%%%%%%%%%%%%%%%%%%%%%%%%%%%%%%%

\documentclass[12pt,twoside]{article}
\usepackage{graphicx}
\usepackage{mathptm}
\usepackage{times}
\usepackage{makeidx}
\usepackage{ifpdf}
\usepackage{footmisc}
\ifpdf
\usepackage[pdftex,
            pagebackref=true,
            colorlinks=true,
            linkcolor=blue,
            unicode
           ]{hyperref}
\else
\usepackage[ps2pdf,
            pagebackref=true,
            colorlinks=true,
            linkcolor=blue,
            unicode
           ]{hyperref}
\usepackage{pspicture}
\fi
\usepackage{url}
\pagestyle{headings}
\makeindex
\emergencystretch=50pt
\setcounter{tocdepth}{3}
\setcounter{secnumdepth}{3}
\title{ESP32 PWM Half Siding Kit 2023 Updates}
\author{Robert Heller \\ The Country Robot \\ Wendell, MA, USA}
\date{\today}
\begin{document}
\maketitle

This is the replacement section for ``Downloadables and Software Support''.

\setcounter{section}{3}

\section{Downloadables and Software Support}

As shipped, the TTGO-T1 has already had the ESP32-PWMHalfSiding firmware
installed, but if you want to rebuild the code (possibly with customizations)
the code is on GitHub in my ESP32-LCC repo:
\url{https://github.com/RobertPHeller/ESP32-LCC}, in the
ESP32-PWMHalfSidingOpenMRNIDF sub-folder. You will need the Espressif's IoT
Development Framework, version 4.2 or later, available from
\url{https://github.com/espressif/esp-idf}. The ESP32-PWMHalfSidingOpenMRNIDF
also uses Mike Dunston's OpenMRNID module -- be sure to do ``git submodule
update --init'' after cloning https://github.com/RobertPHeller/ESP32-LCC.


\subsection{Building and installing the software}

Once you have installed Espressif's IoT Development Framework and cloned the 
ESP32-LCC repo, you can build the software by cd'ing to the 
ESP32-PWMHalfSidingOpenMRNIDF and in bash running these commands:

\begin{verbatim}
. path-to-Espressifs-IDF/export.sh
idf.py build
\end{verbatim}

You can then use the idf.py's flash command to flash the TTGO-T1.

\subsubsection{WiFi notes}

The default configuration is with WiFi disabled. Using both the CAN interface
\textbf{and} WiFi at the same time on the ESP32 is not recomended. The
processor cannot handle both well. It is possible to enable the WiFi code.
This is done by using the menuconfig command with idf.py and then re-building
the firmware. This adds some additional config options to the CDI relating to 
WiFi.

\subsection{Program Description}

The program consists of a main sketch file, and several support files and
``components'' (libraries).

\subsection{Program Startup notes}

While most of the common node configuration is accessable using the available
CDI configuruation tools (JMRI's PanelPro program or MRR SYS OpenLCB program), 
some of the nodes configuration is in separate non-volitle (flash) memory.  
This includes the node's ID number and some boot options.  As installed the 
node ID is initialized to 05:01:01:01:22:00.  \textbf{You don't really want to 
leave this as the node id!}  There are two ways to set the node id.  One way 
is to use a CDI configuruation tool -- the node ID is exposed as a 
configuration option.  Changes the node id forces a ``factory reset'' on the 
next boot (and when you update the configuration with a CDI configuruation 
tool, the node will reboot).  The other way is during the boot up process.  
When the node boots, it waits on the USB serial connection for 10 seconds for 
any ``keyboard'' response and if it gets a ``keyboard'' response, it enters a 
simple command line loop accepting simple commands (mostly single letters). 
To use this feature you need to connect a a USB cable between the TTGO-T1 and 
a computer (eg a PC or Mac) and then connect to the USB Com port with a 
terminal program (the Arduino IDE Serial Monitor should work).  The commands 
that the boot startup command line loop supports are:

\begin{itemize}
\item \textbf{N} Set the node id.  Follow the ``N'' with a 12 digit hex 
number, with optional periods or colons.  Causes a Factory Reset.
\item \textbf{E} Reset events on boot up.
\item \textbf{F} Force a Factory Reset.
\item \textbf{S} WiFi ssid\footnote{Only useful if WiFi is 
enabled.\label{fn:wifi}}. Enter the ssid after the ``S'', leading and trailing 
spaces are stripped off.
\item \textbf{P} WiFi password\footref{fn:wifi}. Enter the password after the 
``P'', leading and trailing spaces are stripped off.
\item \textbf{H} Hostname prefix\footref{fn:wifi}. Enter the hostname prefix 
after the ``H'', leading and trailing spaces are stripped off. The NODE ID in 
hex is appended.  Hostnames are limited to 32 characters.
\item \textbf{W} Enable WiFi\footref{fn:wifi}. Enter ``YES'', ``NO'', 
``ON'', or ``OFF'' after the ``W''.
\item \textbf{T} Test signal lamps\footnote{Each of the signal lamps is 
flashed for 1 second during boot up.}.
\item \textbf{R} Resume.
\end{itemize}



\end{document}
